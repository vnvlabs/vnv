 {\bf{\underline{Education}}}

  \begin{itemize}
\setlength{\itemsep}{0.2pt}
  \item University of Colorado at Boulder, Boulder, CO
\begin{itemize}
\item Ph.D. in Applied Mathematics, August 2017.
\item Masters in Applied Mathematics, May 2015.
\end{itemize}
\item The University of Waikato, Waikato, NZ
\begin{itemize}
\item Bachelor of Science (Hons) in Mathematics and Physics, December 2011. 
\end{itemize}
  \end{itemize}
  
{\bf {\underline{Selected Research \& Professional Experiences}}}
  
\begin{itemize}
\setlength{\itemsep}{0.2pt}
  
\item\underline{Senior Researcher, RNET Technologies, July 2017 - Present}
  \begin{itemize}
	  \item An Extensible Verification and Validation Library with NEAMS Workbench Integration. DOE Phase I SBIR, Primary Investigator (DE-SC0018728) July 2018 - Present
 \item VERA-Workbench: A unified Multi-physics toolkit for the VERA Suite of Tools.
DOE Phase I SBIR, Primary Investigator (DE-SC0017701) October 2017 - May 2018
%-Developing a unified IDE for the VERA suite of tools with python and Enthought Canopy. -The tool provides a rich graphical interface allowing end-users design, execute and visualize pellet-to-plant nuclear physics simulations in a rich, graphical and integrated environment.
\item Cloud-based Scientific Workbench for Nuclear Reactor Simulation Life Cycle Management.
DOE Phase II SBIR, Research Scientist (DE-SC0015748) October 2017 - Present
%-Developing CloudBench, a web-enabled infrastructure that increases adoption of advanced simulation tools.
%-CloudBench enables identification of modeling codes (often developed by major government projects such as NEAMS and CASL), remote execution %and cloud visualization, and workflow management/provenance.
\item Automatic solver selection for Nuclear Engineering Simulation.
DOE Phase II SBIR, Research Scientist (DE-SC0013869) July 2017 - Present
%- Developing a machine learning based solver selection tool for the NEAMS toolkit.
%- The tool analyzes the sub problem characteristics at runtime and select the optimal solver with minimal overhead based on previously trained machine learning models
%- High prediction accuracy has been demonstrated for optimal linear solver selection in terms of execution time.
\end{itemize}
\item\underline{Graduate Student, University of Colorado at Boulder, April 2014-August 2017}\\\
{\em Mentor: Tom Manteuffel (CU Boulder), Jacob Schroder (Lawrence Livermore National Laboratory)}
\begin{itemize}
\item Studied Parallel-in-time solvers for parabolic partial differential equations
\item Implemented a fully adaptive parallel-in-time solver using the FENICS finite element package, C/C++, and MPI.
\item Designed and implemented a temporal load balancing algorithm for the opensource XBraid project with O(log(P)) communication.
\end{itemize}
\item\underline{Summer Internship, May-August , 2014-2016 Lawrence Livermore National Laboratory}\\
{\em Mentor: Jacob Schroder and Rob Falgout}\\
\begin{itemize}
\item Researched an embedded error estimate for the XBraid project.
\item Designed and implemented a cost efficient parallel-in-time solver based on MGRIT and Richardson Extrapolation.
\end{itemize}
  \end{itemize}
  
{\bf {\underline{Refereed Publications }}}
\begin{itemize}
\setlength{\itemsep}{0.2pt}
\item R. D. Falgout, T. A. Manteuffel, B. O'Neill, and J. B. Schroder, Multigrid reduction in time for nonlinear parabolic problems, Copper Mountain Special Section, SIAM J.Sci. Comput. (accepted), (2016). LLNL-JRNL-692258 .
\item M.T. Wilson, P.A. Robinson, B. O'Neill., D.A. Steyn-Ross, Complementarity of Spike- and and Rate-Based Dynamics of Neural Systems, PLOS Computational Biology, Vol 8 (2012).
   \end{itemize}

