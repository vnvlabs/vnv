\section{Principal Investigator and other Key Personnel}
%{\em The Principal Investigator (PI) must be knowledgeable in all
%  technical aspects of the grant application and be capable of leading
%  the research effort and meet the requirements described in Part III,
%  C.  Describe the effort to be performed by the PI during the
%  project.}

\subsection{Gerald Sabin, PI}
Dr. Gerald Sabin, Project Manager at RNET, is the PI for the
CloudBench project. Dr.  Sabin is a full time employee of RNET, and
has sufficient time to dedicate to project tasks as indicated in the
cost proposal. Since he is a US Citizen, he can undertake relevant
integration work in Export-Controlled areas of the project, if
necessary. Currently, he is working on several Scientific Computing
(HPC) SBIR/STTR projects at RNET. He is the PI for this Phase I SBIR
project (DE-SC0015748) developing a ``Web Infrastructure for Remote
Modeling and Simulation of Nuclear Reactors and Fuel Cycle
Systems''. He is the PI on the ongoing Phase II DOE proposal
(DE-FOA-001490) for the Automated Solver Selection for Nuclear
Engineering Simulations. He has also worked on distributed memory,
GPU, multi-core and SIMD optimizations to the Air Force's Kestrel code
(DOD Contract\#:FA9550-12-C-0028) and is currently involved in
developing fine-grained power profiling hardware and software tools
for HPC application profiling (DOE Contract\#:DE-SC0004510). He has
also been the PI on several other related projects including a NASA
Phase I project developing SIMD optimizations for Monte Carlo codes
(NNX14CA44P), developing parallelization optimizations for PETSc (DOE
Contract\#: DE-SC0002434), developing data virtualization support and
bitmap indexing for massive Climate Modeling data sets (DOE Contract
\#:DE-SC0009520), and developing the SmartNIC software stack for
application-aware network offloading (DOE Contract\#:
DE-FG02-08ER86360).

\subsection{Ramachandran Narayanan}
Mr. Ramachandran Narayanan is proposed as a research scientist for
this project and he will work closely with Dr. Sabin. He will assist
Dr. Sabin in the development, implementation, and
testing. Ramachandran has experience using, developing, and optimizing
CFD solvers while working at ANSYS and at Stone Ridge Technologies,
and while pursuing his Computer Science M.S. degree from Penn State
University, and his B. Tech and M. Tech degrees in Aerospace
Engineering from IIT Madras, India. His work included the development
and optimization of transport equations on a GPU, a stand alone
parallel C/C++ Multi-grid solver, and a modular Python code to
accurately trace streamlines given any 2D velocity field. His combined
Aerospace and High Performance Computing background will greatly
benefit this project.

He is a major software developer for this Phase I ``Web Infrastructure
for Remote Modeling and Simulation of Nuclear Reactors and Fuel Cycle
Systems'' project and on the DOE SBIR Phase II project (DE-SC0011312)
that involves the development of a MapReduce like API for Data
Intensive Processing.
