\section{Principal Investigator and other Key Personnel}
%{\em The Principal Investigator (PI) must be knowledgeable in all
%  technical aspects of the grant application and be capable of leading
%  the research effort and meet the requirements described in Part III,
%  C.  Describe the effort to be performed by the PI during the
%  project.}

\subsection{Ben O'Neill, PI}
Ben O'Neill is a Research Scientist at RNET and will be the PI on this project. Ben is a full time employee at
RNET and has sufficient time to dedicate to this project. Ben is a permanent resident in the USA (citizen of New Zealand). The proposed work does not include any Export Control restriction, as such, this work visa should be sufficient. In addition to the current project, Ben is the lead developer in RNET's
Cloudbench project which aims to develop a web-enabled interface for remote execution and visualization for nuclear physics tools. Ben was also heavily involved in the development of the SolverSelector framework for facilitating automatic linear solver selection in high performance applications through machine learning. His background is in Applied mathematics with a focus on high performance computing and parallel-time integration. His thesis work involved the optimization and implementation of a parallel time integration codes for nonlinear PDEs including implementing a fully adaptive and parallel space-time solver using the Fenics finite element package.

\subsection{Gerald Sabin}
Dr. Gerald Sabin, Project Manager at RNET, will be the senior advisor for this project.
Dr.  Sabin is a full time employee of RNET, and has sufficient time to dedicate to project tasks as indicated in the
cost proposal. Since he is a US Citizen, he can undertake relevant
integration work in Export-Controlled areas of the project, if
necessary. Currently, he is working on several Scientific Computing
(HPC) SBIR/STTR projects at RNET. He is the PI for this Phase II SBIR
Cloudbench project (DE-SC0015748) and the ongoing Phase II DOE SBIR for the Automated Solver Selection for Nuclear
Engineering Simulations. He has also worked on distributed memory,
GPU, multi-core and SIMD optimizations to the Air Force's Kestrel code
(DOD Contract\#:FA9550-12-C-0028) and has also been involved in
developing fine-grained power profiling hardware and software tools
for HPC application profiling (DOE Contract\#:DE-SC0004510). He has
also been the PI on several other related projects including a NASA
Phase I project developing SIMD optimizations for Monte Carlo codes
(NNX14CA44P), developing parallelization optimizations for PETSc (DOE
Contract\#: DE-SC0002434), developing data virtualization support and
bitmap indexing for massive Climate Modeling data sets (DOE Contract
\#:DE-SC0009520), and developing the SmartNIC software stack for
application-aware network offloading (DOE Contract\#:
DE-FG02-08ER86360).
