
\section{Related Work}
\label{related}

RNET and ORNL have past and current experience in several SBIR/STTR projects on modeling and simulation, high performance computing, 
and large data formats. Some of these projects are briefly described below. 

\subsection{RNET's Related Work}

\subsubsection{Automated Solver Selection for NEAMS Tools}

RNET in collaboration with University of Oregon (Dr. Boyana Norris) is developing an add-in feature for the NEAMS 
toolkit being developed by the Department of Energy. This work is being done as part of DOE Phase II STTR project 
(Contract Number DE-SC0013869). This add-in feature being developed by RNET leverages machine learning techniques 
to automatically select the optimal solver based on run-time dependent features of the problem and the underlying 
compute architecture with minimal runtime overhead in solver selection during the course of NEAMS simulations. 


\subsubsection{Cloud-based Scientific Workbench for Nuclear Reactor Simulation Life Cycle Management}
The predictive modeling approaches and softwares being continually developed and updated by the DOE nuclear engineering scientists (under programs such as NEAMS, CASL, RISMC etc.) need to be efficiently transferred to the nuclear science and engineering community. An advanced workflow management workbench is required to allow efficient usage from small and large business and research groups. The workbench must manage inputs decks, simulation execution (on a local machine, a High Performance Compute cluster, or a Cloud cluster), intermediate results, final results and visualizations, and provenance of the tools and settings. CloudBench is a hosted simulation environment for large scale numeric simulations. CloudBench will augment existing simulation, Integrated Development Environment, and workbench tools being developed by the DOE and industry. It offers a complete set of simulation management features not available in open tools: sharing of configurations, simulation output, and provenance on a per simulation or per project basis; multi-simulation provenance history to allow simulations to be reconstructed, verified, or extended; and remote access to simulation tools installed on Cloud and HPC resources. The portal enables easy adoption of government codes. 

%% \subsubsection{Catalytic Converter Modeling on High-End Workstations}

%% RNET Technologies, Inc. in collaboration with Prof. Sandip Mazumder developed performance optimizations for accurate 
%% CFD simulation of full-scale monolithic catalytic converters as an alternative to extrapolating single channel simulation 
%% to the entire monolith \cite{Choudary1}. This project has been funded by Department of Energy under the STTR program 
%% (Contract Number DE-SC 0007580). A numerical method (and code), recently developed at OSU, that successfully demonstrated 
%% simulation of laboratory-scale catalytic converters with realistic surface chemistry has been revamped and optimized for 
%% industrial-scale simulations. The enhancements made to the existing method included rewriting certain functionalities to 
%% multicore processors and GPGPUs. Using these performance optimizations, a speedup of ~4.5X has been achieved for 3D test 
%% problems with 300K cells, 20+ reactions, and various mesh topologies. These optimizations will facilitate Catalytic 
%% Converter Simulation on high-end workstations and small clusters by fully utilizing the compute resources (Multicores, 
%% GPUs, vector processing units, etc.) in emerging architectures. 

\subsubsection{Scaling the PETSc Numerical Library to Petascale Architectures}

RNET has developed an extended version of the numerical library PETSc \cite{Lowell1} in
collaboration with Ohio State University and Argonne National Lab. PETSc is an MPI-based numerical library of
linear and nonlinear solvers that is widely used in a variety of scientific domains. With the
emergence of multicore processors and heterogeneous accelerators as the building blocks of
parallel systems, it is essential to restructure the PETSc code to effectively exploit multi-level
parallelism. Changes to the underlying PETSc data structures are required to leverage the multicore
nodes and GPGPUs being added to the ``cluster architectures''.

This project was funded by Department of Energy under the STTR program from August 2010 (Contract Number DE-SC0002434) 
to May 2013. Dr. P. Sadayappan (OSU) and Dr. Boyana Norris (ANL) have played a key role in this effort by serving as 
technical advisors. As part of the project, the team has investigated ways for the PETSc library to fully utilize the 
computing power of future Petascale computers. Novel sparse matrix types,  vector types, and preconditioning techniques 
that are conducive for GPU processing and SIMD parallelization have been integrated into the PETSc library. The matrix 
vector operations have been optimized for specific architectures and GPUs by utilizing the autotuning tools.


\subsubsection{Optimization of Kestrel for Emerging Architectures}
RNET and OSU are performing this work as part of an active DOD Phase
II STTR (FA9550-12-C-0028, Highly-Scalable Computational-Based
Engineering). Based on the identification of the main performance
bottlenecks in the Kestrel CFD code (based on the AVUS CFD solver), we
are developing enhancements to improve the performance of the kCFD
solver, as well as interface other scalable Krylov subspace sparse
solver libraries to Kestrel. The proposed work will address the
effective exploitation of parallelism at multiple levels: SIMD/SIMT
level, multi-core level, and multi-node level.

As part of this project CUDA kernels are also being explored for the
bottlenecks in the CFD and CSD solvers. For instance, a GPU-based CFD
solver with an identical interface to the current Block-Seidel solver
is being explored.



\subsection{ORNL's Related Work}
%\subsubsection{ICE}
\label{sec:nice}

Dr. Watson is the main developer for the Eclipse Parallel Tools Platform (PTP). The aim of the PTP project is to produce an open-source industry-strength platform that provides a highly integrated environment specifically designed for parallel application development. The platform provides a standard, portable parallel IDE that supports a wide range of parallel architectures and runtime systems; a scalable parallel debugger; support for the integration of a wide range of parallel tools; and an environment that simplifies the end-user interaction with parallel systems. 

In addition, Dr. Watson has played a significant role in the development of GAURD, a parallel relative debugger for high performance systems. Unlike other conventional parallel debuggers a relative debugger provides the
ability to dynamically compare data between two executing programs regardless of their location and configuration. In GAURD, data
comparisons can be performed either using an imperative scheme or a declarative scheme. Imperative comparisons can be performed
explicitly by the user when two programs under the control of the debugger are stopped at breakpoints. 

A solid foundation has been laid to achieve the objectives of the proposed project. The final 
 piece of the puzzle is the optimization and hardening of these core approaches for efficient execution in real applications. 
 Once this is accomplished, it will be possible to seamlessly integrate functionality for end-user \VV into any general purpose 
 numerical simulation software. The collaboration between ORNL and 
RNET is a perfect fit to facilitate this final step. RNET brings to the table the required computer science expertise to 
satisfy the needs of this project, as is evident from a description of their related work, and ORNL brings a wealth of experience developing novel
HPC software solutions, developing robust debugging tools for use in HPC environments and working with a range of real application codes. 
