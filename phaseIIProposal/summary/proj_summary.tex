
\section*{Project Information}
Company Name: RNET Technologies\\
Title: VnV: A Self Documenting Testing Framework for In-situ Verification and Validation in High Performance Computing Applications. \\
Principal Investigator: Dr. Ben O'Neill\\
Topic Number/subtopic letter: 30d (Modeling and Simulation)

\section*{Problem Statement}
%%Statement of the problem or situation that is being addressed. Describe the problem or situation being 
%%addressed . be sure that the DOE interest in the problem is clear, but not in such a way that implies that 
%%any service or products are being provided for the direct benefit of DOE rather than for the 
%%advancement of a public purpose. (Typically one to three sentences).
Verification and Validation (V\&V) of numerical simulations is a discrete process that cannot realistically account for every possibility. This causes issues in general purpose simulation packages because, while it may be the responsibility of the developers to ensure that their product is mathematically correct, it is the responsibility of the end-user to ensure the solution obtained is a suitable representation of their physical model. After all, the direct costs of a design failure, be it time, money or loss of life, fall squarely on the shoulders on the end-user, and any attempt to shift the blame to the developers of simulation library $X$ will certainly fall of deaf ears. 
\section*{General Statement}
%%General statement of how this problem is being addressed. This is the overall objective of the combined 
%%Phase I and Phase II projects. How is this problem being addressed? What is the overall approach of the 
%%combined Phase I/Phase II project? (Typically one to two sentences).
The VnV toolkit facilitates the development of \emph{explainable} numerical simulations that, in addtion to the final solution, provide 
the end-user with a detailed report on why the solution can be trusted. These reports provide the detailed knowledge
required to robustly verify and validate end-user driven simulations.

\section*{Phase I Feasibility}
%%What was done in Phase I (and Phase II, if applying for Sequential A or B)? (Typically two to three 
%%sentences).
The Phase effort demonstrated the feasibility of developing a framework that facilitates end-user V\&V
in advanced numerical simulations. This included the development of 
a portable solution for declaring injection points in a code, an intuitive interface 
for writing tests and injecting them at runtime, and an automated documentation generation
system with supoprt for 2D and 3D visualization. 

\section*{Phase II Plans}
%%What is planned for the Phase II project? (Typically two to three sentences).
The Phase II effort will develop the complete VnV framework, including the development
of custom preprocessor directives that allow injection points to be configured in more dynamic ways 
and efficient statistical metrics for asserting the state of data 
stored in distributed arrays. The value of the VnV framework will be demonstrated using the NEAMS 
toolchain of MOOSE, PETSc and libMesh. 

\section*{Commercial Applications and Other Benefits}
%%Commercial Applications and Other Benefits (limited to the space provided). Summarize the future 
%%applications or public benefits if the project is carried over into Phase III and beyond. Do not repeat 
%%information already provided above.
Numerical modeling and simulation (M\&S) is almost always more economical than live prototyping; 
a fact that has seen wide-scale uptake of M\&S in industry (e.g., automotive, nuclear, 
aerospace, advanced manufacturing, etc.). In all cases, the explainable numerical simulations facilitated
by the VnV toolkit provide end-users with the wealth of knowledge required to ensure dangerous errors
do not propagate into final designs. 

\section*{Key Words}
%%Provide listing of key words that describe this effort.
Verification, Validation, Accredidation, In-situ testing,
HPC, Documentation generation, ADIOS, NEAMS.

\section*{Summary for Members of Congress}
%% (layman.s terms, two sentences with a maximum of 50 words). The 
%%DOE notifies members of Congress of awards; therefore, please provide, in clear and concise layman.s 
%%terms, a very brief summary of the project, suitable for a possible press release from a Congressional 
%%office.
The VnV framework provides scientists with the tools required 
to create explainable numerical simulations that, in addition to the final report, provide
users with a detailed report on why the solution should be trusted. Such information will
help end-users ensure uncaught simulation errors do not propagate into final designs. 
