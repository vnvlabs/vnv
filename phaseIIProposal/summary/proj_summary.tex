
\section*{Project Information}
Company Name: RNET Technologies\\
Research Institute: Oak Ridge National Laboratory\\
Title: Cloud-based Scientific Workbench for Nuclear Reactor Simulation  Life Cycle Management\\
Principal Investigator: Dr. Gerald Sabin\\
Research Institute PI: Dr. Jay Billings\\
Topic Number/subtopic letter: 30d (Modeling and Simulation)

\section*{Problem Statement}
%%Statement of the problem or situation that is being addressed. Describe the problem or situation being 
%%addressed . be sure that the DOE interest in the problem is clear, but not in such a way that implies that 
%%any service or products are being provided for the direct benefit of DOE rather than for the 
%%advancement of a public purpose. (Typically one to three sentences).
The predictive modeling approaches and softwares being continually
developed and updated by the DOE nuclear engineering scientists (under
programs such as NEAMS, CASL, RISMC etc.) need to be efficiently
transferred to the nuclear science and engineering community.  An
advanced workflow management workbench is required to allow efficient
usage from small and large bossiness and research groups. The
workbench must manage inputs decks, simulation execution (on a local
machine, a High Performance Compute cluster, or a Cloud cluster),
intermediate results, final results and visualizations, and provenance
of the tools and settings.

\section*{General Statement}
%%General statement of how this problem is being addressed. This is the overall objective of the combined 
%%Phase I and Phase II projects. How is this problem being addressed? What is the overall approach of the 
%%combined Phase I/Phase II project? (Typically one to two sentences).
CloudBench is a hosted simulation environment for large scale numeric
simulations.  CloudBench will augment existing simulation, Integrated
Development Environment, and workbench tools being developed by the
DOE and industry. It offers a complete set of simulation management
features not available in open tools: sharing of configurations,
simulation output, and provenance on a per simulation or per project
basis; multi-simulation provenance history to allow simulations to be
reconstructed, verified, or extended; and remote access to simulation
tools installed on Cloud and HPC resources.

\section*{Phase I Feasibility}
%%What was done in Phase I (and Phase II, if applying for Sequential A or B)? (Typically two to three 
%%sentences).
The Phase effort demonstrated the feasibility of using existing DOE
investments (i.e., the ICE workbench) to launch and manage remote jobs
on cluster and commercial cloud software. In addition, an initial
native web interface was demonstrated for the proposed
workbench. These two pillars will support the development of the full
web based large scale simulation workbench, with an initial focus on
open source and government sourced Nuclear Engineering applications.

\section*{Phase II Plans}
%%What is planned for the Phase II project? (Typically two to three sentences).
The Phase II effort will develop the complete CloudBench:NE tools
including a web interface for workflow management and user provenance
sharing, remote job launch and management (support DOE clusters, DOE
private clouds, and major public clouds), and
visualization. CloudBench:NE will initially support major NEAMS and
CASL simulation codes. Additional codes (within the Nuclear
Engineering community and in other large scale numeric simulation
verticals) will be supported with non-SBIR funds.


\section*{Commercial Applications and Other Benefits}
%%Commercial Applications and Other Benefits (limited to the space provided). Summarize the future 
%%applications or public benefits if the project is carried over into Phase III and beyond. Do not repeat 
%%information already provided above.
Many engineers use vendor supported computer aided simulation tools,
which include vendor supported workbench and job management
tools. Many open source and government sources codes provide
simulation features not available in these commercial codes. However,
in order for engineers to adopt advanced simulation tools there must
be an open workbench that supports easy deployment, job execution, and
sharing. CloudBench will provide these features, allowing users to
adopt the advanced simulation tools.

\section*{Key Words}
%%Provide listing of key words that describe this effort.
Remote Job Management, Simulation Workflows, Provenance, Life Cycle
Management, Large Scale Numerical Simulations, NEAMS, CASL, Web Based
Workbench, Cloud, Clusters, HPC 



\section*{Summary for Members of Congress}
%% (layman.s terms, two sentences with a maximum of 50 words). The 
%%DOE notifies members of Congress of awards; therefore, please provide, in clear and concise layman.s 
%%terms, a very brief summary of the project, suitable for a possible press release from a Congressional 
%%office.
CloudBench is web based advanced simulation management tool to support
open- and government- sourced applications. It will enable
identification of modeling codes (often developed by major government
projects such as NEAMS and CASL), remote execution and cloud
visualization, and workflow management/provenance. The portal enables
easy adoption of government codes.
