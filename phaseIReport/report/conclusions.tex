\section{Summary and Conclusions}

During Phase I of the project, RNET and ORNL have tackled various technical 
challenges with regard to developing a modern interface for enabling Cloud 
Computing for Scientific Computing tools. This capability, when fully 
developed, will provide a convenient workbench-like interface for automatically 
setting up various computing tools on cloud-based clusters as well as 
supporting sharing and provenance-based workflows. In order to demonstrate 
feasibility of developing this product, RNET and ORNL have accomplished the 
following.

\begin{itemize}
\item Prototyped a web-based UI for CloudBench using Vaadin and a system for 
choosing users to share data with and verifying appropriate notifications.
\item Resolved various technical challenges in isolating the Core, producing a 
standalone executable and setting it up to run as a Remote OSGi service, which 
is crucial towards being able to run on any server.
\item Validated the Remote OSGi connection and the ability to connect via an 
EDEF File.
\item Ported user interface components like a Nek5000 form to Vaadin and in 
future, will port other forms from ICE.
\end{itemize}

The above evaluations demonstrate the technical feasibility of the proposed 
approach, initially for CloudBench and eventually for CloudBench:NE. This 
platform also has great commercial potential. RNET intends to promote 
CloudBench by initially supporting relevant open-source computational tools to 
leverage their existing customer base. There is also a great interest from 
NEAMS community and the broader computational community who will benefit from 
this approach.


\section{Publications and Presentations}

The PI for the project has attended the NEAMS annual review meeting from 
December 1st - 3rd, 2015 at ORNL in Oakridge, TN. The meeting presented an 
opportunity to meet the entire NEAMS community and get a comprehensive 
understanding of the NEAMS program. The PI has presented a poster at the 
meeting, discussed the project with the NEAMS community, and made
professional connections that resulted in a collaboration for Phase II.