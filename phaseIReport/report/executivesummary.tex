\section*{Executive Summary}
This SBIR project is aimed at developing CloudBench, a workbench-like tool that 
will allow automatic setup and execution of computational codes onto cloud 
based, compute or local clusters, along with workflow based provenance support 
and sharing of computed results. Our initial focus will be on open source 
Scientific Computing tools such as Nek5000. CloudBench:NE will be the 
specialization of CloudBench for NEAMS tools, many of which are closed source 
tools and need specific licenses.

Scientific simulations are growing in complexity, ranging from climate 
simulations to understanding the interactions and structures of various 
proteins and their role in diseases. This causes an ever increasing demand for 
high-performance hardware which most Users cannot simply afford to stay ahead 
of. Many of these tools also incur an overhead in building and setting up an 
environment where they can be used. Also, scientists and researchers work 
together in a collaborative environment and need to have methods to 
conveniently share and reproduce results. CloudBench aims to resolve these 
issues by providing support for automatic setup of these tools, an intutive 
interface to cloud-based (or local) clusters and provenance and sharing support 
for results.

During Phase I of the project, RNET and ORNL have worked on various steps in 
the development of CloudBench. This report documents these efforts and 
demonstrates the technical feasibility of the proposed approach. RNET will add 
support for the relevant NEAMS tools that will greatly benefit from this 
approach as well as to leverage the customer base.
%There is significant interest in the NEAMS community for a solution that 
%allows provenance based workflows and sharing of results to accurately 
%reproduce results.

