\section{Identification and Significance of the Problem or Opportunity, and Technical Approach}
%% {\em Define the specific technical problem or opportunity addressed by
%%  your application.  Provide enough background information so that the
%%  importance of the problem/opportunity is clear.  Indicate the overall
%%  technical approach to the problem/opportunity and the part that the
%%  proposed research plays in providing needed results.}

This is the final report for contract DE-SC0015748, a Phase I DOE SBIR
entitled ``Web Infrastructure for Remote Modeling and Simulation of Nuclear 
Reactors and Fuel Cycle Systems''. The work was performed by RNET Technologies 
Inc. (RNET) and Mr. Jay Billings from the Oak Ride National Laboratory (ORNL).

\subsection{Significance and Identification}
\label{intro} 
Modeling and simulation softwares and data sets for nuclear engineering 
applications play an important role in 
the design and operation of nuclear reactors. The Nuclear Energy Advanced 
Modeling and Simulation (NEAMS) program 
by DOE is developing predictive models for the advanced nuclear reactor and 
fuel cycle systems using leading edge 
computational methods and high performance computing technologies \cite{NEAMS}. 
Though the toolkit is mainly targeted 
for DOE nuclear energy R\&D programs and leadership supercomputers, an 
important objective of the NEAMS program is 
to enable widespread use among the industry, academia, and regulatory 
communities\cite{NEAMS}. The predictive 
modeling approaches and softwares being continually developed and updated by 
the DOE nuclear engineering scientists 
(under programs such as NEAMS, RISMC etc.) need to be efficiently transferred 
to the scientific community at large 
including academicians, industrial researchers, and regulatory communities. 

An optimal utilization of this nation's intellectual property in nuclear 
engineering softwares can be enabled through
robust web-based communication and collaboration channels that facilitate code 
and data sharing, remote execution and visualization in a 
cloud, and research collaboration between the seemingly 
diverse communities spread across the 
universities and national labs, industry, and other government organizations. 
Novel web-based platforms are desired to 
relatively accelerate sharing of information (codes, data, computational 
approaches), promote collaboration among interested parties (scientists and 
researchers), and impose de-facto standards for sharing the large variety of 
softwares and methodologies. In 
addition, the Department of Energy has a vested 
interest and has invested heavily in M\&S tools over the last few decades. A 
commercial opportunity exists for small 
businesses in enabling economically feasible yet specialized web-based 
communication and collaboration channels among the 
nuclear science and engineering community and extending it to other scientific 
communities (High Energy Physics, 
Computational Fluid Dynamics, Financial Modeling etc.).

The NEAMS and other nuclear engineering softwares developed by the basic energy 
scientists are spread across custom 
websites and various source code hosting websites (Github, SourceForge, DOE 
labs' websites etc.). A lot of effort is 
required from the end user in downloading, reading the documentation, 
compiling, generating input datasets, executing 
the code, and visualizing the results. Almost all of these codes are 
computationally intensive and require 
state-of-the-artcomputing infrastructure such as small clusters, 
supercomputers, or cloud resources to yield meaningful 
results in reasonable timeframes. These softwares, especially the advanced 
codes such as NEAMS tools are continually 
enhanced with new features and capabilities. For example, a contact 
verification problem in BISON might have been 
enhanced with a new nonlinear solver and preconditioner and mapped to parallel 
compute resources such as GPGPUs. With the growing availability of cloud 
computing services, it makes sense to be able to spin up high-end hardware 
configurations to carry out simulations whenever local resources are exhausted.

%There exists a clear need for a platform that will allow the automatic code 
%download, setup and installation, track changes in a workflow with a 
%provenance 
%model and allow sharing of such workflows easily among various collaborators. 

In order to address the above shortcomings, in this SBIR/STTR project, RNET and 
ORNL are proposing to
develop CloudBench, a web-based framework that facilitates relevant code 
identification, automatic remote execution (on cloud-based or local clusters) 
and visualization, tracking changes in a workflow with a provenance based 
model, and sharing of results between various collaborators. The proposed 
portal is prototyped using Vaadin~\cite{vaadin}, which allows web-based 
interfaces to built and work across various platforms. The cloud computing 
support will enable easy use of various computational tools and include the 
ability to post-process remotely.

Phase I demonstration includes an initial prototyping of the web interface with 
Vaadin and leveraging the Integrated Computational Environment (ICE) modular 
design to separate a back-end or ``Core''. The following sections demonstrate 
the Phase I experiments and studies. 
