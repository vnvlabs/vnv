%% \textbf{Project Narrative} %(Field 7 on the form)
%% {\em The project narrative must be no longer than 20 pages of text.
%%   It must be typed in 12-point, with 1 inch margins.  All grant
%%   applications must be submitted in response to a specific technical
%%   topic and subtopic announced in this notice.  This information
%%   should be identified in a header on each page of the project
%%   narrative as well as on the SF 424 R\&R in the title field (number
%%   11).

%%   Grant applications, submitted to DOE under SBIR/STTR programs, must
%%   provide sufficient information to convince DOE, and members of the
%%   research community who review the grant application, that the
%%   application is responsive to the topic and subtopic under which it
%%   is submitted, that the proposed work represents a sound approach to
%%   the investigation of an important scientific or engineering
%%   question, and that it is worthy of support under the stated
%%   criteria.  The Phase I grant application should describe
%%   self-contained research that will contribute to proving scientific
%%   or technical feasibility of the approach or concept.  It should be
%%   written with the care and thoroughness accorded papers for
%%   publication--direct, concise, informative, and free from grammar,
%%   typographical, and spelling errors.  Illustrations and charts should
%%   be clearly labeled and correctly referenced in the text. Promotional
%%   and non-project-related discussion detracts from the professional
%%   quality of the proposal.  The work proposed for Phase I, assuming
%%   that it proceeds successfully, should be suitable in nature for
%%   subsequent progression to Phases II and III.

%%   Technical reviewers will base their conclusions only on information
%%   contained in the grant application.  Do not assume that reviewers
%%   are acquainted with the small business, key individuals, or any
%%   theory or experiments referred to, but not described.  (This
%%   includes material in refereed professional journals--those in which
%%   the articles have been subjected to peer review, and material
%%   referenced on Internet Web pages).  Relevant journal articles should
%%   be summarized in the grant application.  Information provided via
%%   Web links will not be reviewed.

%%   Specifically excluded from this funding notice are grant
%%   applications principally for literature surveys, for compilations of
%%   the work of others, for technical assessments, or for technical
%%   status surveys.  If any of these types of tasks are included in the
%%   work plan, the grant (if awarded) may be reduced in proportion to
%%   that effort.  In addition, grant applications primarily for the
%%   development of already proven concepts will be declined, because
%%   such efforts are considered the responsibility of the private
%%   sector.

%%   Narrative descriptions of the technical topics are provided.  Each
%%   technical topic is subdivided into a maximum of 4 subtopics,
%%   designated by the letters a, b, c, or d.  A grant application must
%%   respond to a specific technical topic and, within it, to only one
%%   subtopic.  NOTE: The topic numbers change each year.  Be sure to
%%   identify the correct topic number on the SF 424 R\&R in the title
%%   field (number 11).  The application will be evaluated under the
%%   topic number identified.  The DOE will not be responsible for
%%   reassigning applications to the correct topic number if identified
%%   incorrectly.

%%   The Project Narrative format should follow the outline below:}

{
% uncomment this to remove TOC link color.
%\hypersetup{hidelinks}
\tableofcontents
}
\section{Significance, Background, and Technical Approach}

\subsection{Identification and Significance of the Innovation}
\label{sec:identification}
%% {\em Define the specific technical problem or opportunity addressed by
%%  your application.  Provide enough background information so that the
%%  importance of the problem/opportunity is clear.  Indicate the overall
%%  technical approach to the problem/opportunity and the part that the
%%  proposed research plays in providing needed results.}

RNET Technologies Inc. (RNET) in Dayton, OH and Oak Ridge National Laboratory 
(ORNL) are responding to 2017 DOE SBIR/STTR Phase II Release 2 
(DE-FOA-0001646). This proposal is for a Phase II contract in succession to an 
initial Phase I contract (Contract \#: DE-SC0015748) awarded for topic DOE 
SBIR/STTR Topic 30d (Modeling and Simulation). RNET has extensive SBIR 
experience in various aspects of High Performance Computing such as performance 
optimization of numerical softwares and libraries, development of fine-grained 
power monitoring tools for HPC infrastructure, and large scale data analysis 
tools. RNET is currently developing a machine learning based plugin for 
automated solver selection in NEAMS tools. Mr. Billings (ORNL) has extensive 
experience with NEAMS tools. He was the Principal Investigator in NEAMS for the NEAMS
Integrated Computational Environment (NiCE), now known as the Eclipse Integrated Computational Environment (ICE), for eight years. This project has successfully transitioned to other sponsors and funding mechanism and enjoys broad recognition outside the DOE research community as an official Eclipse.org project. Mr. Billings also leads the Eclipse Advanced Visualization Project which was originally developed to support multiple visualization tools for NEAMS, and he also worked on numerous other projects in NEAMS, including the initial development of Warthog, the development of the SHARP build system, and interactions with the Advanced Reaction Concepts team. He now leads the Scientific Software Development team at ORNL and, among other projects, is redeveloping the Department of Energy's Energy Science and Technology Software Center to support large, interdisciplinary codes like those from the NEAMS community. Therefore, this team is well positioned to develop and 
commercialize a cloud based platform workflow and provenance management tool for nuclear engineering simulations. 

\subsubsection{Identification and Significance}
\label{intro}
Modeling and simulation software and data sets for nuclear
engineering applications play an important role in the design and
operation of nuclear reactors. The DOE Nuclear Energy Advanced
Modeling and Simulation (NEAMS) program is developing predictive
models for the advanced nuclear reactor and fuel cycle systems using
leading edge computational methods and high performance computing
technologies \cite{NEAMS}. In addition to the role the NEAMS tools
play in the DOE nuclear energy R\&D programs, an important objective
of the NEAMS program is to enable widespread use among the industry,
academia, and regulatory communities\cite{NEAMS}. The predictive
modeling approaches and softwares being continually developed and
updated by the DOE nuclear engineering scientists (under programs such
as NEAMS, RISMC, CASL etc.) need to be efficiently transferred to the
scientific community at large including academicians, industrial
researchers, and regulatory communities. The transfer of these tools
to industry users can facilitate the development of next generation
nuclear reactors.

An optimal utilization of this nation's intellectual property in
nuclear engineering softwares can be enabled through robust web-based
communication and collaboration channels that facilitate code and data
sharing, remote execution in clouds and High Performance Computing
(HPC) clusters, workflow and provenance management and sharing, and
remote web/cloud visualization in order to support research
collaboration between the seemingly diverse communities spread across
the universities and national labs, industry, and other government
organizations. Novel web-based platforms are desired to relatively
accelerate sharing of information (codes, data, computational
approaches), conglomerate the interested parties (nuclear engineers,
applied mathematicians, software developers, and end users), promote
collaboration, and impose de-facto standards for sharing the large
variety of softwares and methodologies. In addition, the Department of
Energy has a vested interest and has invested heavily in M\&S tools
over the last few decades. A commercial opportunity exists for small
businesses in enabling economically feasible yet specialized web-based
communication and collaboration channels among the nuclear science and
engineering community and extending it to other scientific communities
(High Energy Physics, Computational Fluid Dynamics, Financial Modeling
etc.). Vendor based Computer Aided Engineering (i.e., numerical
simulation) platforms generally include a workflow and life cycle
management component, which is crucial to allowing users to easily
install and utilize the CAE tools. Unfortunately, the vendor tools only
support vendor specific applications. This encourages vendor ``lock
in'' and does not facilitate the integration of advanced government
or open source codes, or the integration of ``best-in-breed'' tools
across multiple vendors.

The NEAMS toolkit and other nuclear engineering software developed by the basic energy scientists are spread across custom 
websites and various source code hosting websites (Github, SourceForge, DOE labs' websites etc.). A lot of effort is 
required from the end user in downloading, reading the documentation, compiling, generating input datasets, executing 
the code, and visualizing the results. Almost all of these codes are computationally intensive and require 
state-of-the-art-computing infrastructure such as small clusters, supercomputers, or cloud resources to yield meaningful 
results in reasonable timeframes. These software, especially the advanced codes such as the NEAMS tools are continually 
enhanced with new features and capabilities. For example, a contact verification problem in BISON might have been 
enhanced with a new nonlinear solver and preconditioner and mapped to parallel compute resources such as GPGPUs. 

%% In addition, there does not exist a common platform that connects the nuclear engineering community and is 
%% driven by its users. Various interest groups exist on social networking websites (LinkedIn, Facebook, 
%% ResearchGate etc.) that are active and vibrant but additional channels are required to connect the basic 
%% energy scientists with the scattered end user groups. In order to enable wide spread adoption of nuclear
%% energy modeling tools, it is imperative to provide such state-of-the-art web services under one common 
%% platform and facilitate higher level semantic web applications. For example, using a single web-based 
%% platform, the researcher should be able to search for a nuclear engineering code, download the input 
%% dataset, run the simulations in cloud, visualize the results, share the results (both publicly or privately)
%% on a professional social networking platform. 

In order to address these needs RNET Technologies Inc. (RNET) and Oak
Ridge National Laboratory (ORNL) are developing CloudBench. Our
product is a numerical simulation workbench that assists the user in
setting up, executing, and visualizing advanced simulation
applications and workflows. CloudBench workflows will be dynamic and with support ``human in the loop'' research needs. CloudBench supports remote execution (on Cloud resources
and HPC Clusters) and local job execution. Users will have the
capability to share simulation input, output, and simulation
provenance with authorized users. The initial iteration,
CloudBench:NE, will focus on advanced Nuclear Engineering
applications. Additional versions will be developed to support other
Large Scale Numerical Simulation users. CloudBench will be available
in a hosted environment (i.e., SaaS) or as a private server. This
will facilitate an easy to use hosted service or a private, closed
internal service based on customer needs. CloudBench will enable easy use
of NEAMS and other nuclear engineering softwares, rapid dissemination
of research results, easy sharing of knowledge, and a quick feedback
mechanism.


%% Phase I demonstration includes initial prototyping of the web portal and remote execution. A subset 
%% of NEAMS tools will be used to demonstrate execution of nuclear simulations on remote compute 
%% infrastructure yet local visualization. In addition, the cloud services that meet the government 
%% regulatory and compliance requirements (NASA Nebula, AWS GovCloud etc.) will be identified. The Phase 
%% II effort will fully develop the portal including integrating the complete set of NEAMS tools and other 
%% nuclear engineering softwares of interest, supporting multiple computing platforms (cloud facilities, 
%% super computing infrastructure, small clusters, and high end workstations), investigating other higher 
%% level semantic web operations, and developing the software indirections and abstractions that are necessary 
%% to extend the prototype to other research fields. In Phase III, the portal and the techniques will be 
%% extended to other scientific domains such as metereology and weather forecasting, industrial CFD, and 
%% high energy physics. The targeted customers include nuclear energy companies, NASA divisions, DOD and 
%% its Prime Contractors (e.g., Boeing, Lockheed) etc.

\subsubsection{Product Overview and Technical Approach}

CloudBench is a hosted simulation environment for large scale numeric
simulations. CloudBench:NE is the application of CloudBench to Nuclear
Engineering simulation tools. The initial focus is on advanced reactor
tools, but CloudBench:NE will include support for advanced reactors
and light-water reactors simulation tools. CloudBench will augment
existing simulation, Integrated Development Environment, and workbench
tools being developed by the DOE and industry. It offers a complete
set of simulation management features for open source, government
sourced, and commercial simulation codes in a single integrated
workbench;

\begin{itemize}
\item interactive workflow management using a ``human-in-the-loop'' approach,
\item sharing of configurations, simulation output, and provenance on
  a per simulation or per project basis (ensuring that export control
  and license restricts are maintained),
\item hosted versions of advanced simulation codes (removing the need
  for the end user to perform the installation),
\item multi-simulation provenance history to allow simulations to be
  reconstructed, verified, or extended, and
\item remote access to simulation tools installed on Cloud and HPC
  resources.
\end{itemize}

Cloud Bench will allow the users to setup, launch/execute, and
visualize their simulations through a web-based workflow management
interface. The workflow management will allow outputs to be reused
(possibly after the output is translated and/or remeshed using
provided/built in or user generated/provided scripts) as input for
another simulation. References to the inputs, outputs, applications,
and workflow will be recorded. These records will provide provenance
for the set of simulations and allow the original results to be
regenerated for verification. The provenance also provides a known set
of working experiments that can be modified to support further
research. A user will be able to modify or extend a set of
experiments, and rerun the simulations to formulate new results.

The workflow management includes the ability to remotely manage job
execution and records with sufficient detail to provide provenance for the
simulation results. The provenance will include the simulation tools
used (including version and build information where applicable),
simulation parameters, input files, operating systems (and
environment), and hardware. The provenance provides
sufficient information to allow the simulations to be verified, but
also provides a basis for verifying the results using different
environments and inputs. This improves both the scientific
significance and the regulatory acceptance for nuclear simulation
results.

The provenance (input, output, and application information) can be
selectively shared with other users. This sharing will increase
collaboration and allow for regulatory agencies to get a precise
provenance on the applications, hardware, settings, and inputs.

In addition to provenance and data sharing, CloudBench provides a
front-end for third-party simulation tools and supports simulation
execution on local and remote (Cloud, HPC) resources.  This provides
access to preinstalled simulation tools without the burden of
installing the tools. Additionally, it provides easy access to public
and private HPC resources that are available to the user.

CloudBench is a scientific workflow framework that will increase the
usability, access, and value of numerical simulation tools. The online
CloudBench framework will be licensed to small/medium research sized
groups (start up companies, small government or academic labs, etc.)
on a seat-by-seat basis. The CloudBench server will also be licensed
to large organization for installation on private resources. This will
allow the organization to independently control access to simulation
software and datasets by hosting the server on internal
resources.

\subsection{Anticipated Public Benefits}
CloudBench users include businesses and other institutions (e.g.,
government research labs, universities, energy companies) that perform
large-scale numerical modeling and simulation using high performance
compute infrastructure (large super computers, small clusters, and
high-end workstations). The companies and government labs in the
numerical software development business (e.g., ANSYS, Cd-adapco, NEAMS
toolkit, CASL) continuously develop new tools that would benefit from
CloudBench.
%The businesses in cloud and high performance computing
%(e.g., NVidia, Intel, Amazon, Google) develop novel hardware and
%compute environments that must be supported by these applications and
%understood by users.
The benefits of CloudBench include the adoption of
new advanced simulation tools to improve product development and
product design.

%% CloudBench will facilitate wide adoption of high performance and
%% advanced government simulation codes (including NEAMS and associated
%% tools) by the academia, industry, and regulatory communities.  The
%% extreme benefit of this technology to the Nuclear Engineering
%% Community can not be overstated. A serious issue with the simulation
%% tools such as the NEAMS Toolkit is that its technology
%% \textbf{\textit{is too new to run on existing vendor and utility
%%     hardware}}. To wit, the NEAMS Toolkit is state of the art and many
%% industry tools can only be used on much older hardware and operating
%% systems which makes it impossible to compile the NEAMS
%% Toolkit. Usability (including workflow and provenance management) and
%% code access are limiting factors to adoption by third-party commercial
%% users.

The CloudBench differentiating factor is support for advanced open
source tools. The existing workbench and life cycle management tools
offered by traditional vendors support only the vendor's tools, and do
not include support for advanced open source simulation tools. Analysts
like Frost and Sullivan \footnote{``Global CAD and Modeling Software Market,'' Frost \& Sullivan, January 2013. [subscription required].} expect that efficient workflows
and access to codes will drive the adoption of next generation
simulation technology. Workflow and life cycle management tools are
not readily available for the state-of-the-art open source or
government license tools (e.g., the NEAMS and CASL). Therefore,
CloudBench will fulfill this role and provide much needed support for
these advanced simulation tools.

The initial release of CloudBench is CloudBench:NE (for Nuclear
Engineering) will be developed during Phase II. NE is an important
first niche market.  As the U.S. and the world ramp up to deploy new
advanced nuclear reactors, the ability to leverage advanced codes,
deploy on HPC and Cloud resources, and share experiments (for
collaborative and regulatory scenarios) is becoming critical to the
design and regulatory process. While the required workflow management
tools often exist in vendor simulation suites, they are missing from
open sourced and government codes.

This initial NE market includes United States companies such as GE
Hitachi, Westinghouse Electric Company, AREVA, Anatech, Nuscale,
Bechtel Marine Propulsion Corporation, Tennessee Valley Authority,
Studsvik Scandpower, Terrapower, Oklo, Starcore Nuclear,
FPoliSolutions, BWXT Technologies, Transatomic Power Corporation, X
Energy, Terrestrial Energy, Areva, and Flibe Energy.

\subsection{Phase I and Feasibility Demonstration}
\rnetprop{
RNET worked closely with Mr. Jay Billings from ORNL and the lead architect and 
principal investigator for the Eclipse Integrated Computational Environment 
(ICE) in prototyping CloudBench. ICE can interface to many different 
computational codes and NEAMS tools and its modular design allows us to 
separate various components in preparation for a web-based interface. Our 
approach has been to separate ICE into a front-end and back-end service (which 
would run on any compute, cloud or local instance). The front-end User 
Interface (UI) would be reimplemented using a web-based UI framework like 
Vaadin~\cite{vaadin}. This can connect to the back-end which will then allow 
simulations to be remotely executed. This requires the use of Remote OSGi 
services~\cite{osgi_ecf} which automatically distributes and proxies 
communication correctly. The back-end service is also referred to 
as the ``Core'', since it is developed from ICE's Core component.
}

In order to demonstrate feasibility of developing this product, RNET and ORNL 
have accomplished the following.

\begin{itemize}
\item \rnetprop{Prototyped a web-based UI for CloudBench using Vaadin and a system for 
choosing users to share data with and verifying appropriate notifications.}
\item \rnetprop{Resolved various technical challenges in isolating the Core, producing a 
standalone executable and setting it up to run as a Remote OSGi service, which 
is crucial towards being able to run on any server.}
\item \rnetprop{Validated the Remote OSGi connection and the ability to connect via an
EDEF File.}
\item \rnetprop{Ported user interface components for a Nek5000 form to the Vaadin 
interface.}
\end{itemize}

The following subsections provide details on the efforts to
demonstrate CloudBench feasibility.

\subsubsection{Validation of Remote Connection}
\rnetprop{
The Core component in ICE was setup to be a Remote OSGi service. The setup to 
test this is outlined in the Phase I Feasibility Report and is shown here in 
Figure~\ref{fig:test_remote_connections}.
}

\begin{figure}[thb]
\begin{center}
\leavevmode
\includegraphics[width=0.5\linewidth]{./narrative/figures/ice_core_connect.png}
\end{center}
\caption{Setup to test Remote Connections.}
\label{fig:test_remote_connections}
\end{figure}

%%REMOVE FOR SPACE
%% The remote connection is confirmed by the ImportRegistration event on the Host 
%% Windows machine as shown in Figure~\ref{fig:edef_connect}.

%% \begin{figure}[thb]
%% \begin{center}
%% \leavevmode
%% \includegraphics[width=0.7\linewidth]{./narrative/figures/edef_connect_cropped.png}
%% \end{center}
%% \caption{Import Registration on Host Machine.}
%% \label{fig:edef_connect}
%% \end{figure}

\subsubsection{Development of a CloudBench UI}
\rnetprop{Vaadin was chosen as the framework to develop a web-based interface. This was 
mainly due to the intuitive APIs Vaadin offers, the excellent Vaadin security model, as well as the automatic 
adjustments to the interface, based on the device being used to view it, such 
as smartphones and tablets~\cite{vaadin}. Upon successful user login, the 
CloudBench Dashboard is presented where the main area contains a panel with 
up-to-date relevant information on status of executed jobs 
%(\rnetcomment{Can add some panels here})
and notifications on shared output.}
%(\rnetcomment{Currently, only notification, no file sharing yet}).

\begin{figure}[!htb]
\begin{center}
\leavevmode
\includegraphics[width=0.7\linewidth]{./narrative/figures/cloudbench_dashboard_cropped.png}
\end{center}
\caption{CloudBench Dashboard.}
\label{fig:cbench_dashboard}
\end{figure}

\subsubsection{Simple Sharing and Notifications}
\rnetprop{We have designed CloudBench to support File Sharing among a group of Users. To 
facilitate this, the UI includes an ``search as you type'' box to pick out 
Users to share data with.}

\begin{figure}[!thb]
\begin{center}
\leavevmode
\includegraphics[width=0.7\linewidth]{./narrative/figures/cloudbench_sharedialog.png}
\end{center}
\caption{Selecting Users to Share With.}
\label{fig:share_users}
\end{figure}

In this case, we are sharing data with two test Users, ``Solomon Olsen'' and 
``Elvis Olsen''. When we login to one of their accounts, we can see that a 
notification from the originator (test User, ``Gabrielle Patel'') is visible 
and informs the User of the shared object.

\begin{figure}[!thb]
\begin{center}
\leavevmode
\includegraphics[width=0.9\linewidth]{./narrative/figures/eolsen_shared.png}
\end{center}
\caption{Share Notification for Chosen User.}
\label{fig:share_eolsen}
\end{figure}

\subsubsection{Headless Workflow Execution and Remote Job Execution}
\label{sec:remote_exec}
\rnetprop{ Mr. Billings and the ICE team worked with RNET personnel to modify the workflow 
engine in Eclipse ICE such that it can execute workflows headlessly as a Remote 
OSGi service. Although the ICE Core could be run headlessly before, it was by a 
different mechanism that did not meet the needs of the RNET team. Specifically, 
Remote OSGi services greatly simplify both memory management and communications 
between the service client and service provider while maintaining all of the 
normal advantages of an OSGi service that the original mechanism - a purely 
RESTful web service - lost. One other advantage of using the Remote OSGi 
service is that service discovery is simplified on small networks, which has 
great advantages in production deployment. This achievement simplifies workflow 
processing in CloudBench by removing the dependency on the ICE workbench, and 
RESTful service to provide a much simpler programming API. One major advantage 
of this is that workflow execution and job launch can now be performed on a 
server that communicates remotely with the CloudBench web client, while still 
maintaining all of the normal remote job execution capabilities in ICE. ICE's 
normal job launch framework supports local and remote job execution for several 
codes in NEAMS and offers full support for batch systems such as SLURM and PBS 
as well as parallel performance monitoring and remote debugging tools.}


\rnetprop{These updates were used to develop a working example of Vaadin with OSGi and 
implemented the Nek5000 parameters form in Vaadin as shown in 
Figure~\ref{fig:vaddin_nek5000}. This example demonstrates the feasibility of 
leveraging ICE's existing support for NEAMS tools in an easy, extendible way to 
support CloudBench.}

\begin{figure}[!thb]
\begin{center}
\leavevmode
\includegraphics[width=0.8\linewidth]{./narrative/figures/vaadin_nek5000_cropped.png}
\end{center}
\caption{Vaadin implementation of the Nek5000 form.}
\label{fig:vaddin_nek5000}
\end{figure}

\subsubsection{Advanced Visualization}
\label{id_advViz}
\rnetprop{In addition to the Phase I efforts, existing visualization work performed by the ICE team will be used in CloudBench and demonstrate feasibility of the proposed approach.
The Eclipse Advanced Visualization Project (EAVP) was originally developed as part of the NEAMS program to provide visualization support for 3D post-processing visualizations using VisIt and Paraview in tandem with ICE's workflow engine as well as to provide support for 2D graphing, and 3D mesh and geometry editing. This project was spun off separately from ICE several years ago to answer requests from users to use it outside of ICE and it has continued to grow on its own ever since. The project provides}
\begin{itemize}
\item \rnetprop{VisIt integration for working with mesh data}
\item \rnetprop{ParaView integration for working with mesh data}
\item \rnetprop{2D mesh editing support (for Nek5000)}
\item \rnetprop{3D mesh editing support (for surface meshes)}
\item \rnetprop{3D geometry editing}
\item \rnetprop{A full plotting library that supports remote updates and streaming}
\item \rnetprop{Support for remote renderers and multiple connections across VisIt and Paraview}
\end{itemize}




\section{The Phase II Project}
\label{sec:phaseII}
%% {\em Provide an explicit, detailed description of the Phase I research
%%   approach and work to be performed.  Indicate what will be done, by
%%   whom (small business, subcontractors, research institution, or
%%   consultants), where it will be done, and how the work will be
%%   carried out.  If applicant is making a commercial or in-kind
%%   contribution to the project, please describe in detail here.  The
%%   Phase I effort should attempt to determine the technical feasibility
%%   of the proposed concept which, if successful, would provide a firm
%%   basis for the Phase II grant application.

%%   Relate the work plan to the objectives of the proposed project.
%%   Discuss the methods planned to achieve each objective or task
%%   explicitly and in detail.  This section should be a substantial
%%   portion of the total grant application.} 

\subsection{Technical Objectives}
%% {\em State the specific technical objectives of the Phase I effort,
%%   including the questions it will try to answer to determine the
%%   feasibility of the proposed approach.}

The requirements being addressed include the development of a robust framework 
for in-situ verification and validation in general purpose numerical simulation 
packages. In particular, the objective of this project is to address the need
for tools that automate verification of end-user numerical solutions in the 
NEAMS toolkit and workbench. 

In Phase II, RNET Technologies and ORNL will pursue the following objectives:

\begin{enumerate}

\item \rnetprop{Harden and extend the core VnV functionality developed during phase I. In particular, the phase II effort 
will look to determine the optimal approach for implementing the run time configurable test injection system such that 
the risks associated with memory corruption and constant correctness are minimized. This objective will include the miscilanious tasks
required to prepare the framework for release, including integration with a unit testing framework, futher development of the documentation generation 
engine, and documentation. }
\item \rnetprop{Develop mechanisms for efficient data movement in a distributed environment. This objective will 
look to determine and implement optimal approaches for comparing data stored in distributed arrays against an expected 
result stored on disk. The key issue here is to define approaches for describing the domain decomposition of the distributed 
array such that the experimental data can be distributed in an efficient manor. In-situ
comparison of variables with experimental and/or analytical results will be a defining feature of the framework because it significantly 
reduces the amount of IO required in \VV testing, while also providing a fine grained mechanism for detecting at what point a solution diverges from
the expected result.}
\item \rnetprop{Optimize test execution times. Initial development will focus on mechanisms for offloading tests to an external server. Offloading of tests to an
external VnV testing service has the potential to significantly reduce overall runtime. This key issue to address here 
will be to develop a mechanism for offloading data such that the data transfer is faster than simply running the test in-situ. The initial focus will be 
on determining the best framework for offloading simple tests (MRNet, SNOBall, ADIOS streaming, etc.). After that has been implemented, the focus will shift
toward implementing job based parallelism for tests the require the simulation to be run multiple times. }
\item \rnetprop{ Develop a robust set of generic \VV tests. The development of these tests (e.g., mesh refinement studies, the method of manufactured solutions, sensitivity analysis, uncertainty quantification) will further equip the users with the tools required to robustly perform end-user \VV.  The open research question the Phase II project will look to address is the optimal approach to integrating existing implementations of the tools (i.e., NIMROD, DAKOTA, MASA) into the in-situ \VV testing framework. Another open question is the development of a generic interface for mesh refinement studies such that we can automatically generate the required grid hierarchy.}
\item \rnetprop{Demonstrate the value of the VnV framework as a component in the NEAMS tools and into the NEAMS workbench. The true benefit of  
the VnV toolkit will only be realized if we can drive wide scale uptake across the entire numerical simulation community. By showing the toolkit can be used 
in the NEAMS tools, and in particular, MOOSE, libMesh and PETSc, we will demonstrate true potential of the product in libraries that are already considered 
cutting edge across the industry. Integration into the NEAMS tools will answer the NEAMS call for tools that support end-user verification
of numerical simulations. Integration into the NEAMS workbench will provide access to the tools in the seamless manor users of the workbench have come to expect. } 

\end{enumerate}

\subsection{Work Plan}
\label{sec:workplan}

As described in the objectives, the final deliverable of the Phase II project will be a fully functional, effcient, battle tested framework 
for integrating advanced end-user verification and validation into general purpose numerical simulation packages. In what follows, we outline
the work required to satisfy the objectives outlined in the previous section and to acheive these goals. 

\subsubsection{Hardening and Optimization of the Injection Point System}

The injection point system developed during Phase I uses C style Macros and string based (void*) pointer casts to declare the variables available for inspection
at each injection point.  When implemented correctly, this is an efficient (void* casts are almost free) and portable (it uses low level C functionality supported by all C/C++ compilers) approach. In Phase II, the project team will develop a custom compiler that supports an annotation based specification for the injection points and the injection point variables. This annotation based specification will be designed to address two key issues with the Phase I approach:

\begin{itemize}
 \item {\bf String based pointer casts:} The injection point system developed in Phase I requires the developer to provide a string that describes the type for each variable available at an injection point. Under the hood, the injection point system using string comparisons to ensure compatibility been test parameters and injection point variables. The key issue with this approach is that the strings specified at each injection point cannot be verified during compilation.  This will causes major issues in cases where, say, the developer changes the type of a variable, but forgets to update the type string in the injection point. The custom source code processor developed in Phase II will automatically detect the type of the variables passed to the injection point system, removing the requirement for a hard-coded type string. 
  
 \item {\bf Restrictive type specifications:} The current system uses C compliant pre-processor macros to simplify the process of describing injection points and variables. The benefit of this approach is that the injection points can be compiled into any application without significant changes to the build system. The downside is that single-pass, text based macros place a significant restriction on the functionality that can be implemented. The annotation system developed in Phase II will be far more dynamic, allowing users to full control over what variables can be accessed at each injection point, including the ability to provide access to internal components of data structures, describe the domain decomposition of distributed arrays and to complete pre-test processing of variables. The annotation system will also provide a mechanism for suggesting default tests to be run at each injection point, and will provide a mechanism for injection point detection in non-object orientated programming languages where runtime injection point detection cannot be completed.  
\end{itemize}

This compiler will be written using Clang. Clang is a compiler front end for the C family of languages. It is a well supported, well documented compiler package designed as a drop in replacement for GCC. In particular, Clangs simple API for defining custom Pragma routines, combined with its robust API for walking the Abstract syntax tree of the code make it the perfect choice for developing this annotation based system for defining injection points. Clang has seen wide scale uptake across the software development world and, in recent years, has emerged as a realistic competitor to the ubiquitous GCC compiler suite.  

Figure~\ref{TODO} shows how the annotation based injection point system will look in a C/C++ code. The user will annotate simple variables with the ``@Variable(''IP\_1``, ''IP\_2``,...)'' annotation. In that annotation, users will be able to specify the injection points for which that variable is available. Additionally, users will be able to specify additional parameters that define how the variable should be accessed, e,g., for providing access to a member of a struct (line TODO), or for providing access to a certain element in an array (line TODO). For distributed arrays, the annotation based system will allow users to define approaches for obtaining the global and local ownership for each processor. Together, this annotation based system will allow users to fully control the access to these variables. The original Phase I approach will still be available for users that cannot switch to the new Clang compiler. Where possible, the custom compiler will make use of available C++11 RTTI information (dynamic\_cast, typeid, etc.); however, a general C approach will be favored due to the large number of C programs still regularly used in scientific computing. 


\subsubsection{Efficient Statistical Comparisons in Distributed Settings} 

One of the major benefits of the VnV framework is that it allows for in-situ testing and analysis in distributed systems. In particular, this type of in-situ analysis is particularly useful for analyzing and comparing data stored in distributed arrays. For example, a tests could be used to assert that all the elements in an array are positive, or that all the elements in an array are the same as the elements in another array; all without ever writing the data to disk and/or without collecting the data on a head node. 

To that end, the Phase II project will research efficient techniques for analyzing, asserting and comparing data stored in distributed arrays. To achieve this, the project team will need to address two key issues; determining the data decomposition and efficiently performing the statistical analysis. 

{\bf Data decomposition:} In order for the testing algorithms to act of the data structures, some information about the global distribution of the data will be required. Determining this global partition is not difficult for programs that use a regular decomposition scheme because, in these cases, there exists a static mapping function that can be evaluated locally on any processor to determine the owner of any index. For programs using irregular decompositions, no such mapping exists. Instead, for irregular domains, it is necessary to calculate the global partition dynamically at runtime. Hence, a major goal of the Phase II project will be to implement an efficient, scalable approach to dynamically obtaining the global partition from generic distributed arrays. 

To do this, the project team will investigate several approaches to determining intra-processor communication patterns for distributed data. 

The first approach implemented will be to generate the global partition through a collective MPI communication (MPI\_AllGatherV). In this setting, the global partition is assembled on every processor. This makes determining the owner of a particular data component a simple task. 

The above approach requires O(P) storage on each processor and O(log(P)) communication costs (where P is the number of processors). For small processor counts, this linear dependence on P for storage costs is not likely to be a problem. However, in large-scale settings this will quickly become an issue \cite{hypre-assumed}. In particular, such storage costs are an issue for VnV testing because they will be required on top of the storage costs associated with the application being tested. 

To address these storage concerns, the project team will implement an Assumed partition algorithm. This algorithm was first introduced in \cite{} as an efficient mechanism for determining intra-processor communication patterns in Hypre, a scalable collection of high performance linear solvers. The assumed partition algorithm is a parallel rendezvous algorithm that facilitates intra-processor communication using O(1) storage and O(log(P)) communication. The algorithm can be characterized by the following three steps;
\begin{itemize}
 \item Step 1: Assume the global distribution of data 
 \item Step 2: Redistribute the actual partition description to the assumed partition
 \item Step 3: Use the assumed partition to determine global information
\end{itemize}

The key to the assumed partition algorithm is that it assumes the data is distributed using a regular distribution that 
can be queried locally on each processor using a O(1) function call. Step 2 of the algorithm, the data required to form a distributed directory
containing information about the actual data distribution is communicated to the assumed partition. Once this distributed directory has 
been formed, it can be directly queried to determine the owner of the required data in the actual distribution. In this way, we obtain a full realization of the global partition using O(1) storage on each processor (each processor stores information regarding the owner of each index in the local ownership range on the assumed partition). The Phase II implementation of the assumed partition algorithm will support structured grids. Alternative approaches for supporting irregular decompositions in unstructured grids will be investigated in Phase III. 

{\bf Efficient Statistical Analysis:} In a high performance setting, it is simply to collect the data to a root processor for processing; both because of this high communication costs and because, in many cases, the arrays are simply to large to fit in the memory of a single core. Thus, there is a significant need for efficient methods for performing statistical analysis on distributed arrays in a distributed setting. 

To address this need, the Phase II effort will include the development of a statistical VnV testing library for distributed arrays. Like all other VnV testing libraries, users will be able to configure these tests to run at injection points using the VnV configuration file. The testing library will include a number of simple statistical metrics, including the mean, mode, median, standard deviation, co-variance, etc. A collection of tools for verifying physical properties of matrices will also be included (e.g., matrix norms, diagonal dominance, symmetry, sparsity patterns, positive definite-ness, etc.) 

The major research effort required to build this statistical testing suite will be the implementation of efficient approaches for comparing the data in distributed arrays with an expected result. In the context of \VV, such comparisons are required for regression testing, for benchmark testing, and when comparing results to experimental data. 

TODO 

\subsubsection{Reducing Run-times with Test Offloading and Job Parallelism}

Performing a large number of \VV tests in a distributed environment will be expensive, both computationally and due to the data movement required to deal with the domain decomposition employed by the application. A key goal of the Phase II effort will be to investigate and implement approaches for distributing tests for offloading tests for execution on separate processes. 

The issue of efficient data transfer will be addressed using the ADIOS 2 Sustainable Staging Transport (SST). The SST is a classic streaming data architecture, that allows for direct connection between data produces and data consumers via the ADIOS2 write/read API. In HPC environments, SST uses the RDMA interconnects to ensure fast transfer of data between HPC applications; however, socket based connections are also supported. Due to issues associated with serialization of generic data-structures, the Phase II implementation will restrict data offloading to tests that work with the basic data types supported by ADIOS2 (strings, floats, double, arrays, etc). Generic data structures will be investigated in Phase III. The development of test offloading will proceed in two stages; 

The first step will be to develop the interfaces and annotations required to transfer data to external processes for testing. Test offloading will only be effective in situations where the cost of completing the required tests is large in comparison to the time required for transferring the data. To address this issue, the project team will implement a heuristic algorithms that determines the appropriate action based on the size and type of the data structures to be transferred and the type and number of tests to be executed. More robust approaches will be investigated after testing and analysis of the initial implementation.  
 
The second step will be to develop the separate VV executable that consumes the data and coordinates the execution of the required tests. As shown in Figure~\ref{TODO}, this will a MPI application consisting of one master processor and any number of slave processors. The role of the master processor is to efficiently allocate the incoming tests and data to the slave processors. This will include determining the appropriate number of processors to allocate to each test given the available resources, forming the required communication groups, and executing the tests. The performance of the proposed approach will be tested in a range of high performance applications using codes available to RNET and our collaborators at ORNL (including in MOOSE, PETSc, and libMesh). If the performance of the proposed approach is unacceptable, the project team will investigate the benefits of other frameworks that support similar functionality, including MRNet and SNOBall. 
 
In many cases, it will be more cost efficient to execute the tests as separate jobs. To support this use case, the Phase II effort will also include an investigation into launching tests as separate jobs using other resources. Such a functionality would allow for speedup through job parallelism for expensive testing routines like sensitivity analysis and mesh refinement, while also allowing for specific tests to be run on the optimal architecture (e.g., an test involving image processing could be offloaded to a GPU enabled architecture). The launching of jobs on external resources will be completed using the Eclipse Parallel tools platform (PTP). 

The PTP project provides an integrated development environment to support the development of parallel applications written in C, C++, and Fortran. Eclipse PTP provides; support for the MPI, OpenMP and UPC programming models, as well as OpenSHMEM and OpenACC Support for a wide range of batch systems and runtime systems, including PBS/Torque, LoadLeveler, GridEngine, Parallel Environment, Open MPI, and MPICH2. The PTP platform also include a scalable parallel debugger  and support for the integration of a wide range of parallel tools. In this case, the project team will use the PTP platforms extensive support for integrating with remote job schedulers to launch tests on the best available resource. Once integrated, users will be able to specify the resources available for test offloading as part of the configuration file. At runtime, the internal test offloading engine will use test based meta data to allocate tests to the appropriate resources.  
 
\subsubsection{Integrate Third Party Tools for Mesh Refinement, UQ and Sensitivity Analysis.}

Mesh refinement studies, Uncertainty quantification and sensitivity analysis are all essential components of a robust \VV regimen. To that
end, the Phase II effort will include the development of \VV tests that integrate third party tools to complete these tests. 

The UQ and SA tests will be developed using DAKOTA. DAKOTA provides a set of black-box tools for performing parameter optimization, UQ 
and SA. DAKOTA provides two interfaces for integrating these tools into applications; a black-box approach that uses preprocessors, postprocessors
and the file system to complete the tests; and library API for direct, hard-coded testing. The Phase II effort will focus on integrating the 
DAKOTA tools through the library API as it allows for the tools to be applied directly to any independent function. To that end, the Phase II 
effort will include the development of an custom VnV testing library for direct integration with the DAKOTA library API. This will include the development
of a flexible interface for setting up and running the DAKOTA tests, and the development of the test specification files for displaying the test results 
in an informative and interactive way. 

Support for Mesh refinement will be developed as part a separate plug-able VnV testing library. The intended functionality of this testing 
suite will be to enable the completion of both uniform and adaptive mesh refinement studies given only the initial coarse mesh. This will require 
the development of methods for marking the mesh for refinement and methods for refining the mesh. 

An additional test for tracking and displaying a detailed provenance history for the simulation will also be developed. This provenance tracking
test will be attached to the main function of the executable and will include functionality for documenting the inputs to the code, the build 
date of the executable, a list of the versions, checksums and build dates of any imported shared libraries, copies of the stdout, stdin and stderr, and, where 
appropriate, copies of all output files. Tracking these details is not difficult, yet, this detailed level of information will significantly increase the validity of the \VV report while also providing a mechanism for detecting when a \VV report becomes out of date due to an upstream software update. 

\subsubsection{Integration with NEAMS tools and the NEAMS workbench} 

The next step in the Phase II project will be to integrate the VnV framework into the NEAMS 
tools. Doing so will provide ample opportunities for testing, while also allowing us to demonstrate the performance of the toolkit in real codes with
real applications. Moreover, this integration will answer the solicitations call for tools that support end-user verification of numerical solutions in the NEAMS tools. 

Integration into NEAMS tools will be a three step process:

\begin{itemize}
 \item The first step will be to insert injection points into key locations in the MOOSE, libMesh and PETSc source codes. The optimal locations for 
 inserting these injection points will be determined after detailed profiling of example codes; however, some obvious options include during each linear 
 solver iteration, during each nonlinear iteration, during finite element matrix construction (if it exists) and inside the function for calculating the action of the Jacobian
 on a vector. 
 \item The second step will be to allow users to configure the VnV testing directly in the MOOSE input file, using the MOOSE input
file syntax. This will provide users of MOOSE with a seamless mechanism for setting up and running tests. This will be completed by
developing a custom MOOSE ``Action'' that processes the input at runtime to setup the core VnV testing functionality. 
\item The third step will be to enable context-aware auto-complete for MOOSE based VnV configuration files in the NEAMS workbench. The workbench
has integrated support for extracting the information required to setup input file verification and auto-complete from MOOSE applications; however, there
will likely be some additional work required to correctly setup the auto-complete and verification for tests stored in external libraries. In particular, the current system 
requires that the tests adhere to a specific XSD specification for configuration, but there is not yet a system for extracting the exact parameters required to inject a specific test. To do this, we will create a lightweight executable that loads external test libraries and extracts the appropriate parameter specifications in the ``SON'' format supported by workbench.  
\item The last step will be the development of an interface for editing the YAML specification files used to generate the final report and an interface 
for viewing said final reports. The GUI will be written using standard HTML and Javascript and then integrated into the workbench using the QT QWebEngineView component (QWebView is the workbench is using QT 4). The QWebEngineView component allows for interactions with the core QT window (i.e., open, save, copy, paste, etc.) however, it is important to note that this approach will not allow for a truly ``native'' experience within the workbench. The decision to develop in JS and HTML rather than native QT was made to ensure the final product, the VnV framework, is consistent, and can be used in applications outside of the NEAMS lines of tools. Every effort will be made to ensure the GUI conforms to the NEAMS workbench standards and specifications. 
\end{itemize}

\subsubsection{Hardening of the Automated Report Generation System} 

Another step in the Phase II project will be the extension and completion of the automated report generation system. This will include:
\begin{itemize} 
 \item Adding support for a wider range of data visualization techniques. As shown in the feasibility section, the Phase I prototype supports some 2D plots and 3D plotting of .VTP files. In Phase II we will extend this support to include a wide range of charts and a wide range of VTK file formats. In addition, the project team will add support for rendering 3D plots using an instance of Paraview running on the local machine of the user who is reading the report. This functionality, which is already supported by ParaviewWeb, will allow for more detailed visualizations that can be implemented with static data and post-processing alone. 
 \item Adding support for the industry specified \VV report Templates, such as the DoD \VV template specification \cite{}. This functionality will furter increase the value of toolkit in industry settings where specific formats a required for \VV reporting. 

 \end{itemize}


\subsection{Performance Schedule and Task Plan}
\label{sec:taskplan}

% Use wrapfigure here instead?
\begin{wrapfigure}{r}{0.5\linewidth}%[thb]
%\begin{figure}[thb]
\begin{center}
\leavevmode
%\includegraphics[width=1.0\linewidth]{./narrative/figures/tasks.pdf}
\end{center}
\caption{Overview of task dependencies and timeline.}
\label{fig:tasks}
\end{wrapfigure}

RNET would like to present the project ideas and research plan to the
DOE Program Manager and other interested scientists. The meeting will
be used to discuss features, and identify the specific NEAMS applications and computer
resources that will benefit from this project.  This meeting will be
scheduled soon after the Phase II contract is awarded. The meeting can
be hosted at RNET, a DOE site suggested by the Program Manager or via
a teleconference.

RNET will submit all reports as required by the contract (e.g., annual reports, 
a continuation report, summary reports, and a final report) to the DOE program 
manager and other interested DOE scientists.

The research and development topics described in Section~\ref{sec:workplan} 
will be addressed by the tasks described in the remainder of this section. Most 
tasks require active collaboration between RNET and its collaborators. 
Figure~\ref{fig:tasks} summarizes at a high level the dependencies among tasks  and
approximate anticipated task durations. The duration of the Phase II 
project is 104 weeks. Specific details are included in the description of each 
task.


\newcounter{taskCount}
\setcounter{taskCount}{0}

\refstepcounter{taskCount}\label{task:2.5}
\subsubsection{Task \ref{task:2.5}: Development of the Annotation based Inpjection Point System  }

In this task, the project team will develop the Annotation system for specifying injection points and describing injection variables. This will include the development of
the API for specifying the Annotations as shown in Figure~\ref{TODO}, and the development of the custom LLVM compiler extensions required to process these annotations. At the end of this task, users will be able to specify injection points and injection variables using either the new Annotation point system, or using the more portable, but less robust Phase I approach. 

To test the Annotation based injection point system, the project team will write injection points at several key locations in the MOOSE software stack, including inside the source code of several MOOSE modules, the core MOOSE framework, libMesh and PETSc. This will allow us to demonstrate the cross-library potential of the Annotation based system, while also acting as the first step toward integration of the tools a number of the NEAMS tools. 

\rnetprop{RNET will work on the implementation for this task and ORNL will provide inputs and guidance.}


\refstepcounter{taskCount}\label{task:3}
\subsubsection{Task \ref{task:3}: Develop methods for offloading tests to external processes. }

\refstepcounter{taskCount}\label{task:23}
\subsubsection{Task \ref{task:23}: Development of efficient statistical \VV tools with a focus of performance in large scale distributed settings.}


In this task, the project team will investigate the optimal approaches for conveying information about the data decompoosition to the individual tests. As outlined in Section~\ref{TODO}, this will involve the implementation of the BlockMap algorithm for data structures with regular decompositions and the implementation of a simple approach for generating the global partition for problems with irregular domain decompositions. Based on the performance of those implementations, the project team will investigate more optimal approaches to completing intra-processor communications with distributed data arrays, including creating a data directory or using the assumed partition algorithm 

In addition to this the project team will implement several statistical tools for asserting the state of the data in distributed arrays, as described in Section~\ref{TODO}. 

At the completion of this task, users will be able to efficiently compare and analyze distributed arrays using teh VnV framework. To test these implementations, the project team will set up tests for performing assertions on various PETSc vectors and matrices. The metric for success in this task will be to: (1) support arbitary data decompoistion descriptions; and (2) obtain a factor 5x speedup when performing assertions using our custom statistical assertions when compared to the naieve approach of gathering the data on a single root core for processing. 

\rnetprop{RNET will work on the implementation for this task and ORNL will provide inputs and guidance.}

\refstepcounter{taskCount}\label{task:22}
\subsubsection{Task \ref{task:22}: Development of Generic tools for Mesh refinement, Uncertainty quantification and Sensitivity Analysis.}
In this task, RNET will implement generic \VV tools for mesh refinement, uncertainty quantification and sensitivity analysis. In the case of mesh refinement, the approach taken will be to create a generic interface for interacting with the automatic mesh refinement functionality that already exists in finite element libraries like LibMesh and Fenics. The overall goal is to create a generic VnV test that can be attached to the main function of the executable such that it automates the process of running mesh refinement and mesh convergence studies. In the case of UQ and SA, the project team will develop an interface for specifying the tools available in DAKOTA as VnV tests. 

\rnetprop{RNET will be responsible for this task. ORNL will provide guidance on 
developing the framework for ORNL CADES.}


\refstepcounter{taskCount}\label{task:4}
\subsubsection{Task \ref{task:4}: Extension of the VnV report generation system}
In this task, the project team will complete the development of the VnV report generation system. he primary goal of this task will be
to provide support for generating VnV reports that conform to the specification outlined in the DoD VV report templates.  The task will also include  a full implementation 
of the VnV markdown extension to support a wide variety of data visualization components, the development of the interfaces required for displaying unit and regression testing
reports, and the software indirections required for assimilating multiple VnV reports into a single document that can be used in the \VV of an entire simulation package. 

\rnetprop{RNET will work on the implementation for this task}

\refstepcounter{taskCount}\label{task:1}
\subsubsection{Task \ref{task:1}: Integration into real applications, including the NEAMS Tools}
\rnetprop{
  In this task, the program team will integrate the VnV framework into a variety of real applications. Initially, this 
  testing will be completed in tools used heavily in the NEAMS toolkit; MOOSE, libMesh and PETSc, but other third party 
  applications will also be investigated. To ensure seamless integration with the MOOSE tools, the project team will reimplement
  the XML based configuration file using the MOOSE input file format. This will allow for the configuration of the VnV tests in 
  a MOOSE application directly from the input file.
  
  The goal of this task will be to generate informative, production quality VnV 
  reports for a number of examples available in the MOOSE testing suite. Doing so allows us
  to test every facet of the proposed framework, while also acting the first demonstration of the
  value provided by the framework. These results of these tests will be hosted on the RNET website 
  as they become available. 
 

}

\rnetprop{
RNET will be responsible for this task and ORNL will provide guidance on 
various technical implementations and details.
}
\refstepcounter{taskCount}\label{task:11}
\subsubsection{Task \ref{task:11}: Development of an interface for the NEAMS workbench}
\rnetprop{
  In this task, the project team will integrate the toolkit directly into the NEAMS workbench. This will be
  a two stage process. First, the project team will implement the required interface files for enabling 
  the context aware auto-complete features available in the NEAMS workbench for the MOOSE based configuration
  file specification developed in the previous section. MOOSE based input files are already largely supported in
  the workbench; however, there will likely be some issues with determining which tests are applicable at which 
  injection points. Second will be developing an interface for customizing and viewing the final VV report. As part of the Phase I
  effort, the project team demonstrated viewing the final \VV report in a QT WebView component. The workbench is also but on to
  off QT, hence we do not expect to many difficulties on that front. Instead the key objective will be to develop the mechanisms for
  displaying customizations made the the final report in real-time within the NEAMS workbench. 
}
\rnetprop{ RNET will be responsible for this task }


\subsection{Facilities/Equipment}
\subsubsection{RNET Facilities}
RNET has the necessary office equipment to manage an SBIR/STTR contract
including networks, workstations, and accounting software. In
addition, RNET has the tools (software and hardware) to evaluate and
develop the technologies proposed here.  

RNET currently has 9 development computers and a 10-node development cluster 
that can be used for development and testing in this effort. Each cluster node 
has two quad-core or hexa-core XEON CPUs, 24-32GB of DRAM, 500+GB of local 
disk. 
Two data networks are available, a COTS 1 Gbps Ethernet network and a 10 Gbps 
Ethernet network. The Linux development nodes and the RNET cluster have the 
necessary Linux/GNU toolchains and development environments including; GNU 
tool chain, Microsoft .Net Framework, and Java Standard Edition.

\subsubsection{ORNL Facilities}
%\rnetcomment{Jay to verify, can we state these resources can be used on this project?}
The Oak Ridge National Laboratory (ORNL) hosts three petascale computing 
facilities: the Oak Ridge Leadership Computing 
Facility (OLCF), managed for DOE; the National Institute for Computational 
Sciences (NICS) computing facility operated 
for the National Science Foundation (NSF); and the National Climate-Computing 
Research Center (NCRC), formed as 
collaboration between ORNL and the National Oceanographic and Atmospheric 
Administration (NOAA) to explore a variety of 
research topics in climate sciences. Each of these facilities has a 
professional, experienced operational and engineering 
staff comprising groups in high-performance computing (HPC) operations, 
technology integration, user services, scientific 
computing, and application performance tools.

%\rnetcomment{Ram: Based on Jay's comments.}
ORNL also has the Compute and Data Environment for Science (CADES) which is a 
fully integrated infrastructure offering compute and data services for 
researchers lab-wide. We will work with appropriate program managers to apply 
for allocation requests as appropriate.


 The ORNL computer facility staff 
provides continuous operation of the centers 
and immediate problem resolution. On evenings and weekends, operators provide 
first-line problem resolution for users with 
additional user support and system administrators on-call for more difficult 
problems. ORNL also has state-of-the-art 
visualization facilities that can be used on site or accessed remotely. 

\section{Consultants and Subcontractors}
Oak Ridge National Laboratory(ORNL) is the Research Institution for this proposal and will serve as a subcontractor for this SBIR. 
\subsection{Gregory Watson}
Gregory Watson, PhD, is a Senior Research Scientist in the Computer Science Research Group at Oak Ridge National Laboratory. Dr. Watson's research interests include programming tools and development environments for high performance and scientific computing, software engineering practices, reproducibility, and education and training for scientists. Dr. Watson is the founder of the Eclipse Parallel Tools Platform, a project that was originally started as a collaboration between Los Alamos National Laboratory and IBM in 2004, and that continues to be used across laboratories, academia, and industry. He is also a founding member of the Eclipse Science Working Group, and project leader of the Eclipse Science Top Level Project. Dr. Watson has considerable experience developing and implementing efficient parallel debugging software for high performance computing environments and has a wealth of experience in the development of highly scalable tools and communication frameworks for peta-scale high performance computing systems.

\section{Principal Investigator and other Key Personnel}
%{\em The Principal Investigator (PI) must be knowledgeable in all
%  technical aspects of the grant application and be capable of leading
%  the research effort and meet the requirements described in Part III,
%  C.  Describe the effort to be performed by the PI during the
%  project.}

\subsection{Gerald Sabin, PI}
Dr. Gerald Sabin, Project Manager at RNET, is the PI for the
CloudBench project. Dr.  Sabin is a full time employee of RNET, and
has sufficient time to dedicate to project tasks as indicated in the
cost proposal. Since he is a US Citizen, he can undertake relevant
integration work in Export-Controlled areas of the project, if
necessary. Currently, he is working on several Scientific Computing
(HPC) SBIR/STTR projects at RNET. He is the PI for this Phase I SBIR
project (DE-SC0015748) developing a ``Web Infrastructure for Remote
Modeling and Simulation of Nuclear Reactors and Fuel Cycle
Systems''. He is the PI on the ongoing Phase II DOE proposal
(DE-FOA-001490) for the Automated Solver Selection for Nuclear
Engineering Simulations. He has also worked on distributed memory,
GPU, multi-core and SIMD optimizations to the Air Force's Kestrel code
(DOD Contract\#:FA9550-12-C-0028) and is currently involved in
developing fine-grained power profiling hardware and software tools
for HPC application profiling (DOE Contract\#:DE-SC0004510). He has
also been the PI on several other related projects including a NASA
Phase I project developing SIMD optimizations for Monte Carlo codes
(NNX14CA44P), developing parallelization optimizations for PETSc (DOE
Contract\#: DE-SC0002434), developing data virtualization support and
bitmap indexing for massive Climate Modeling data sets (DOE Contract
\#:DE-SC0009520), and developing the SmartNIC software stack for
application-aware network offloading (DOE Contract\#:
DE-FG02-08ER86360).

\subsection{Ramachandran Narayanan}
Mr. Ramachandran Narayanan is proposed as a research scientist for
this project and he will work closely with Dr. Sabin. He will assist
Dr. Sabin in the development, implementation, and
testing. Ramachandran has experience using, developing, and optimizing
CFD solvers while working at ANSYS and at Stone Ridge Technologies,
and while pursuing his Computer Science M.S. degree from Penn State
University, and his B. Tech and M. Tech degrees in Aerospace
Engineering from IIT Madras, India. His work included the development
and optimization of transport equations on a GPU, a stand alone
parallel C/C++ Multi-grid solver, and a modular Python code to
accurately trace streamlines given any 2D velocity field. His combined
Aerospace and High Performance Computing background will greatly
benefit this project.

He is a major software developer for this Phase I ``Web Infrastructure
for Remote Modeling and Simulation of Nuclear Reactors and Fuel Cycle
Systems'' project and on the DOE SBIR Phase II project (DE-SC0011312)
that involves the development of a MapReduce like API for Data
Intensive Processing.


\section{Related Work}
\label{related}

RNET and ORNL have past and current experience in several SBIR/STTR projects on modeling and simulation, high performance computing, 
and large data formats. Some of these projects are briefly described below. 

\subsection{RNET's Related Work}

\subsubsection{Automated Solver Selection for NEAMS Tools}

RNET in collaboration with University of Oregon (Dr. Boyana Norris) is developing an add-in feature for the NEAMS 
toolkit being developed by the Department of Energy. This work is being done as part of DOE Phase I STTR project 
(Contract Number DE-SC0013869). This add-in feature being developed by RNET will leverage machine learning techniques 
to automatically select the optimal solver based on run-time dependent features of the problem and the underlying 
compute architecture with minimal runtime overhead in solver selection during the course of NEAMS simulations.

%% \subsubsection{Catalytic Converter Modeling on High-End Workstations}

%% RNET Technologies, Inc. in collaboration with Prof. Sandip Mazumder developed performance optimizations for accurate 
%% CFD simulation of full-scale monolithic catalytic converters as an alternative to extrapolating single channel simulation 
%% to the entire monolith \cite{Choudary1}. This project has been funded by Department of Energy under the STTR program 
%% (Contract Number DE-SC 0007580). A numerical method (and code), recently developed at OSU, that successfully demonstrated 
%% simulation of laboratory-scale catalytic converters with realistic surface chemistry has been revamped and optimized for 
%% industrial-scale simulations. The enhancements made to the existing method included rewriting certain functionalities to 
%% multicore processors and GPGPUs. Using these performance optimizations, a speedup of ~4.5X has been achieved for 3D test 
%% problems with 300K cells, 20+ reactions, and various mesh topologies. These optimizations will facilitate Catalytic 
%% Converter Simulation on high-end workstations and small clusters by fully utilizing the compute resources (Multicores, 
%% GPUs, vector processing units, etc.) in emerging architectures. 

\subsubsection{Scaling the PETSc Numerical Library to Petascale Architectures}

RNET has developed an extended version of the numerical library PETSc \cite{Lowell1} in
collaboration with Ohio State University and Argonne National Lab. PETSc is an MPI-based numerical library of
linear and nonlinear solvers that is widely used in a variety of scientific domains. With the
emergence of multicore processors and heterogeneous accelerators as the building blocks of
parallel systems, it is essential to restructure the PETSc code to effectively exploit multi-level
parallelism. Changes to the underlying PETSc data structures are required to leverage the multicore
nodes and GPGPUs being added to the ``cluster architectures''.

This project was funded by Department of Energy under the STTR program from August 2010 (Contract Number DE-SC0002434) 
to May 2013. Dr. P. Sadayappan (OSU) and Dr. Boyana Norris (ANL) have played a key role in this effort by serving as 
technical advisors. As part of the project, the team has investigated ways for the PETSc library to fully utilize the 
computing power of future Petascale computers. Novel sparse matrix types,  vector types, and preconditioning techniques 
that are conducive for GPU processing and SIMD parallelization have been integrated into the PETSc library. The matrix 
vector operations have been optimized for specific architectures and GPUs by utilizing the autotuning tools.


\subsubsection{Optimization of Kestrel for Emerging Architectures}
RNET and OSU are performing this work as part of an active DOD Phase
II STTR (FA9550-12-C-0028, Highly-Scalable Computational-Based
Engineering). Based on the identification of the main performance
bottlenecks in the Kestrel CFD code (based on the AVUS CFD solver), we
are developing enhancements to improve the performance of the kCFD
solver, as well as interface other scalable Krylov subspace sparse
solver libraries to Kestrel. The proposed work will address the
effective exploitation of parallelism at multiple levels: SIMD/SIMT
level, multi-core level, and multi-node level.

As part of this project CUDA kernels are also being explored for the
bottlenecks in the CFD and CSD solvers. For instance, a GPU-based CFD
solver with an identical interface to the current Block-Seidel solver
is being explored.

\subsubsection{A Map-Reduce Like Data-Intensive Processing Framework for Native Data Storage}

%\begin{wrapfigure}{r}{0.5\linewidth}%[thb]
%\begin{center}
%\leavevmode
%\includegraphics[width=1.0\linewidth]{./narrative/figures/mapreduce.png}
%\end{center}
%\caption{iNFORMER: A Native data format MapREDuce-like framework.}
%\label{fig:mr}
%\end{wrapfigure}

RNET is currently under a DOE Phase II STTR contract for developing a MapReduce-like data-intensive processing framework 
for native data storage (Contract\#: DE-SC0011312). The Ohio State University (OSU) is a collaborator on this STTR project. 
MapReduce is a very popular data analytic framework that is widely used in both industry and scientific research. Despite 
the popularity of MapReduce, there are several obstacles to applying it for developing some commercial and scientific data 
analysis applications.


This project is developing a Native data format MapREDuce-like framework, iNFORMER, based on SciMate architecture. The 
framework allows MapReduce-like applications to be executed over data stored in a native data format, without first loading 
the data into the framework. This addresses a major limitation of existing MapReduce-like implementations that require the 
data to be loaded into specialized file systems, e.g., the Hadoop Distributed File System (HDFS). The overheads and additional
data management processes required for this translation can prevent MapReduce from being used in many commercial and 
scientific environments.

% uncomment if including the image.
%Figure \ref{fig:mr} shows how iNFORMER components are related to each other. 
%It also shows how they 
%can relate to other components when integrated with a larger Big Data platform 
%such as Hadoop. 




\subsection{ORNL's Related Work}
%\subsubsection{ICE}
\label{sec:nice}

The Eclipse Integrated Computational Environment (ICE) is an award-winning, open source platform and user environment for working with scientific software, managing scientific 
workflows and exposing advanced modeling and simulation technologies through a streamlined user experience. It includes tools
for input generation, local and remote job launch including remote job monitoring, data analysis and data management. It also
includes software development tools for multiple languages including C/C++, Java and Fortran. It uses community tools for data
analysis including VisIt and Paraview as well as tools developed custom analysis.

It was sponsored continuously by NEAMS from 2009 to 2016, where it was known as the NEAMS Integrated Computational Environment (NiCE), and has extensive 
support for the NEAMS Toolkit. It includes plugins for MOOSE-based and 
SHARP-based applications and also plugins for Warthog, which unifies MOOSE and 
SHARP. It has also received funding from other offices, including the Advanced 
Manufacturing Office, ORNL's Laboratory Director's Research and Development 
(LDRD) fund, and external sponsors. It supports workflows in areas ranging from 
advanced manufacturing, advanced materials, astrophysics, neutron science, 
nuclear energy, quantum computing, and virtual batteries, among others.

ICE's sister project is the Eclipse Advanced Visualization Project, which was also originally part of NEAMS and NiCE. This platform offers a wide range of visualization capabilities and integrates with existing tools in the market, including VisIt and Paraview. It provides either local or remote connections to these tools in addition to offering visualization tools for editing geometry and meshes as well as simple plotting.

Mr. Billings is the project lead and architect for both of these projects.

A solid foundation has been laid to achieve the objectives of the proposed project from the NEAMS perspective. The final 
missing piece of the puzzle is streamlining these efforts to enable cloud execution and a goto-portal for the nuclear 
engineering community. Once this is accomplished, it will be possible to seamlessly transfer all the technology that is 
currently only within DOE and a few selected groups to the broader community at large. The collaboration between ORNL and 
RNET is a perfect fit to facilitate this final step. RNET brings to the table the required computer science expertise to 
satisfy the needs of this project, as is evident from a description of their related work.

