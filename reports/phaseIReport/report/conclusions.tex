\section{Summary and Conclusions}

During Phase I of the project, RNET have tackled various technical 
challenges with regard to developing a modern framework for facilitating 
in-situ end-user verification and validation in general purpose numerical 
simulation packages.  This capability, when fully 
developed, will provide a convenient interface for creating explainable numerical 
simulations that not only provide the user with a solution, but also a detailed 
report on how the solution was obtained and why it should be trusted. 

In order to demonstrate 
feasibility of developing this product, RNET has accomplished the 
following.

\begin{itemize}
 \item Developed cross library support for defining and registering injection points in general purpose numerical simulation packages.
 \item Created an interface for writing and integrating custom V\&V tests that can be configured at runtime.
 \item Developed a custom markdown format that allows for automated post-processing and visualization of testing data.
 \item Implemented an automated documentation generation script that creates a server-less, interactive VnV report that can be 
 displayed in any web browser. 
 \item Demonstrated how the VnV reports could be viewed in the NEAMS workbench through the QWebEngineView. 
\end{itemize}

The above evaluations demonstrate the technical feasibility of the proposed 
approach.  This 
platform also has great commercial potential. RNET intends to promote 
the VnV framework by initially supporting relevant open-source computational tools to 
leverage their existing customer base. 


\section{Publications and Presentations}
None


